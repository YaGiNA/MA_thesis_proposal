\section{Preliminary Results and Discussion}
\subsection{Dataset}
In order to input our proposed model, we obtained FakeNewsNet\cite{shu2017exploiting,Shu:2017:FND:3137597.3137600,shu2018fakenewsnet} dataset.
This includes tweets which has URL of news.
Every news and tweets are labeled true/fake by fact-check result on PolitiFact or GossipCop.
We use tweet text as comments and the other information (user, retweet, like, etc.) are not used.
\reftab{table:fakenewsnet} is statistics of dataset.

\begin{table}[htp]
    \centering
    \caption{Statistics of FakeNewsNet by fact-checking platforms}
    \label{table:fakenewsnet}
    \begin{tabular}{lrrrr}
        \hline 
        & \multicolumn{2}{c}{True} & \multicolumn{2}{c}{Fake} \\
        Platform & News & Comments & News & Comments \\
        \hline \hline 
        PolitiFact & 624 & 370669 & 432 & 195901 \\
        GossipCop & 16817 & 843933 & 5323 & 539491 \\
        \hline 
        Overall & 17441 & 1214602 & 5755 & 735392 \\
        \hline 
    \end{tabular}
\end{table}

\subsection{Plan of experiments}
We will make answer of following evaluation questions:
\begin{description}
    \item[EQ1] Can our proposed model detect fake news more accurate than any other state-of-the-art fake news detection algorithms?
    \item[EQ2] Is generating comments important in fake news detection by Seq-GAN?
    \item[EQ3] Are generated comments similar to real comments?
\end{description}

We are planning to answer them by comparing our model with any other state-of-the-art fake news detection algorithms,
ablation experiments, and subjective evaluation by human beings.

\subsection{Plan of discussion}
\subsubsection{EQ1: comparing}
We will get results of not only our proposed model but also other algorithms which are proposed by related works.
All of them use both of news text and comments for equal comparing.
\subsubsection{EQ2: ablation experiments}
We also compare by ablation experiments.
It does our proposed model with ones which do not use generated comments in order to
testify to importance of generating comments by Seq-GAN.
If proposed model is better than ablated one, generating comments is important part to find fake news.
\subsubsection{EQ3: subjective evaluation}
When training is over, generated comments will be so similar to real comments.
We can measure how far from real comments to generated comments are by subjective evaluation.