\section{Preliminary Results and Discussion}
\subsection{Dataset}
In order to input our proposed model, we obtained FakeNewsNet\cite{shu2017exploiting,shu2017fake,shu2018fakenewsnet} dataset.
This includes tweets which has URL of news.
Every news and tweets are labeled true/fake by fact-check result on PolitiFact or GossipCop.
We use tweet text as comments and the other information(user, retweet, like, etc.) are not used.
\reftab{table:fakenewsnet} is statistics of dataset.

\begin{table}[htp]
    \centering
    \caption{Statistics of FakeNewsNet by fact-checking platforms}
    \label{table:fakenewsnet}
    \begin{tabular}{lcccc}
        \hline 
        & \multicolumn{2}{c}{True} & \multicolumn{2}{c}{Fake} \\
        Platform & News & Comments & News & Comments \\
        \hline \hline 
        PolitiFact & 1 & 1 & 1 & 1 \\
        GossipCop & 1 & 1 & 1 & 1 \\
        \hline 
        Overall & 1 & 1 & 1 & 1 \\
        \hline 
    \end{tabular}
\end{table}

\subsection{Plan of experiments}
We will make answer of following evaluation questions:
\begin{description}
    \item[EQ1] Can our proposed model detect fake news more accurate than any other state-of-the-art fake news detection algorithms?
    \item[EQ2] Is generating comments important in fake news detection by Seq-GAN?
    \item[EQ3] Are generated comments similar to real comments?
\end{description}

We are planning to answer them by corresponding with any other state-of-the-art fake news detection algorithms,
ablation experiments, and subjective evaluation by human beings.