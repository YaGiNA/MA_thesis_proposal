\begin{abstract}
    There are already positive results about classifying between true news and fake news.
    In addition, some works reported considering comments on social media made results better.
    However, comments are not appear when news are just posted and it doesn't suitable for early detection.
    Therefore, there is a previous work which proposed the model which does not only classifying true news or fake news, but also real posted or generated.
    This tried to probably distribution of comment appearance and this achieved generate high quality of classifying in test.
    Our proposed model plan to evolve this approach by generating comments much similar to real ones by Seq-GAN.
    We plan to experiment this model to measure performance of classifying news, importance of generating comments, and quality of generated comments.
    If results will be positive, it shows that generating comments like real posted is helpful to detect fake news.
\end{abstract}
\keywords{Fake News; Seq-GAN; Natural Language Programming; Deep Neural Network; Social Network}