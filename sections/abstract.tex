\begin{abstract}
    Fake news is recognized as a fatal social problem in the past few years.
    There are already some works about classifying between true news and fake news and they required more better classifying results.
    In addition, some works reported considering comments on social media made results improving.
    However, comments are not appeared when news are just posted and it doesn't suitable for early detection.
    Therefore, previous work has been proposed the model which does not only classifying true news or fake news, but also real posted or generated.
    This attempt created a probability distribution of comment appearance and successfully generated high quality of classifying.
    Our proposed model plan to evolve this approach by generating comments much similar to real ones by Seq-GAN.
    We plan to experiment this model to measure performance of classifying news, importance of generating comments, and quality of generated comments.
    If the results are positive, it will show that generating comments like real posted is helpful to detect fake news.
\end{abstract}
\keywords{Fake News; Seq-GAN; Natural Language Programming; Deep Neural Network; Social Network}