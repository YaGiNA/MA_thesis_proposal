\section{Thesis}
The main objective of this research is developing new fake news classification 
with comment generation and investigate how proposed method is better in operation on social media.
To suppress spread of fake news, we have to spot it early enough.
Specifically, it is required classifying before spread of fake news if classifier operate in social media.

In classification of fake news, social contexts give strong information.
Among social contexts, comments give more information as natural language than retweets and likes.
However, it is impossible to get social contexts from news which is just posted on social media.
Therefore, we train model not only classifier but also comment generator for fake news detection.
This use Seq-GAN \cite{Yu:2017:SSG:3298483.3298649} as comment generation with real comments which are posted in Twitter.